% !TEX program = lualatex

\documentclass[11pt,a4paper,sans]{moderncv}

% moderncv theme
% Style options are 'casual' (default), 'classic', 'oldstyle' and 'banking'
\moderncvstyle{classic}
% Color options 'blue' (default), 'orange', 'green', 'red', 'purple', 'grey'
% and 'black'
\moderncvcolor{black}
% To set the default font; use '\sfdefault' for the default sans serif font,
% '\rmdefault' for the default roman one, or any tex font name
%\renewcommand{\familydefault}{\sfdefault}
% Uncomment to suppress automatic page numbering for CVs longer than one page
%\nopagenumbers{}

% Adjust the page margins
\usepackage[scale=0.85]{geometry}
\usepackage{microtype}
% If you want to change the width of the column with the dates
\setlength{\hintscolumnwidth}{2.5cm}
% For the 'classic' style, if you want to force the width allocated to
% your name and avoid line breaks.
% Be careful though, the length is normally calculated to avoid any overlap
% with your personal info; use this at your own typographical risks...
%\setlength{\makecvtitlenamewidth}{10cm}

\usepackage{fontspec}
\setmainfont[
    Ligatures=TeX,
    Scale=MatchLowercase,
    BoldFont=SourceSansPro-Semibold.otf,
    ItalicFont=SourceSansPro-LightIt.otf,
    BoldItalicFont=SourceSansPro-SemiboldIt.otf]{SourceSansPro-Light.otf}
\usepackage{polyglossia}
\setdefaultlanguage{english}
\setotherlanguage{french}

% Default for the "classic" style: marvosym
% "awesome" is the default in the "banking", "casual" and "classic" styles
\moderncvicons{awesome}

% Change the width of the date part of each cventry
\setlength{\hintscolumnwidth}{0.17\textwidth}

% Personal data
\name{Alexis}{Jeandeau}
% Optional
\title{Résumé}
% Optional
% The "postcode city" and "country" arguments can be omitted or provided empty
\address{Heights Nihonbashi Nakasu 307, 6--10 Nihonbashi Nakasu, Chūō-ku}{103--0008 Tokyo}{Japan}
% Optional
\phone[mobile]{+81--90--6791--5920}
% Optional
\email{alexis.jeandeau@gmail.com}

% Social links
\newcommand*{\httpslink}[2][]{%
  \ifthenelse{\equal{#1}{}}%
    {\href{https://#2}{#2}}%
    {\href{https://#2}{#1}}}

% adds a social link to one's personal information (optional)
% usage: \social[<optional type>][<optional url>]{<account name>}
% where <optional type> should be either "linkedin" or "github"
\RenewDocumentCommand{\social}{O{}O{}m}{%
  \ifthenelse{\equal{#2}{}}%
    {%
      \ifthenelse{\equal{#1}{linkedin}}{\collectionadd[linkedin]{socials}{\protect\httpslink[#3]{www.linkedin.com/in/#3}}}{}%
      \ifthenelse{\equal{#1}{github}}{\collectionadd[github]{socials}{\protect\httpslink[#3]{github.com/#3}}}     {}%
    }
    {\collectionadd[#1]{socials}{\protect\httplink[#3]{#2}}}}

\social[github]{jeandeaual}
\social[linkedin]{alexis-jeandeau}

\ifdefined\withphoto%
    % optional
    % '64pt' is the height the picture must be resized to,
    % 0.4pt is the thickness of the frame around it (put it to 0pt for no frame)
    % and 'picture' is the name of the picture file
    \photo[64pt][0.2pt]{photo.jpg}
\fi

\AfterPreamble{\hypersetup{
    % pdfcreator     = {\LaTeX{} with 'moderncv' package},
    pdfcreator     = {\LaTeX{}},
    % pdfproducer    = {\LaTeX{}}, % will/should be set automatically to the correct TeX engine used
    % bookmarksopen  = true,
    % bookmarksdepth = 2, % to show sections and subsections
    pdfauthor      = {Alexis~Jeandeau},
    pdftitle       = {Alexis~Jeandeau -- Résumé},
    pdfsubject     = {Résumé of Alexis~Jeandeau},
    pdfkeywords    = {Alexis~Jeandeau, cv, resume, Go, Erlang, Java, C++}
}}

% Technologies / protocols
\providecommand{\techno}{}
\renewcommand{\techno}{\emph}
% Languages / libraries
\providecommand{\lang}{}
\renewcommand{\lang}{\emph}

%-------------------------------------------------------------------------------
%            content
%-------------------------------------------------------------------------------

\begin{document}

\makecvtitle%

\section{Professional Experience}
%\subsection{Vocational}

\cventry{From September 2019}{Web Engineer}{Tier IV}{Tokyo}{}{%
\begin{itemize}%
\setlength\itemsep{1em}
\item Developed the data platform of an autonomous vehicle fleet management system
    \begin{itemize}%
        \item Responsibilities:
            \begin{itemize}%
                \item Worked on the messaging between a cloud management system and the vehicles
                \item Worked on a fleet management and OTA update system for autonomous vehicles
            \end{itemize}
        \item Technologies used: \techno{\href{https://aws.amazon.com/iot/}{AWS IoT}}, \techno{MQTT}, \techno{Python}, \techno{C++}
    \end{itemize}
\end{itemize}
}

\vspace*{2em}

\cventry{April 2016--August 2019}{Software Engineer}{Z-Works}{Tokyo}{}{%
\begin{itemize}%
\setlength\itemsep{1em}
% ZIO
\item Developed the backend of an IoT system for the elder care market.
    \begin{itemize}%
        \item Responsibilities:
            \begin{itemize}%
                % \item Managed the backend development team (3 members);
                \item Development a microservice infrastructure in \lang{Erlang} and \lang{Java} to manage IoT devices and end users
                \item Creation of a communication API using \techno{AMQP} and \techno{RabbitMQ}
                \item Continuous integration using Bitbucket Pipelines, AWS CodeDeploy and various Amazon services
            \end{itemize}
        \item Technologies used: \techno{AMQP}, \techno{Z-Wave}, \techno{\href{http://www.linear.com/products/smartmesh_ip}{Dust Networks SmartMesh IP}}, \techno{PostgreSQL}, \lang{Erlang}, \lang{Java}
    \end{itemize}
\item Developed an administration interface aimed at support engineers.
    \begin{itemize}%
        \item Responsibilities:
            \begin{itemize}%
                \item Creation of a REST API using Amazon services
            \end{itemize}
        \item Technologies used: \lang{Java}, \href{https://aws.amazon.com/api-gateway/}{Amazon API Gateway}, \href{https://aws.amazon.com/lambda/}{AWS Lambda}
    \end{itemize}
\end{itemize}
}

\cventry{May 2012--April 2016}{Software Engineer}{Takahashi Giken}{Tokyo}{}{%
\begin{itemize}%
\setlength\itemsep{1em}
% LPUS
\item Created a back-end service to allow real-time photo viewing and management during events (sports, wedding, etc.).
    \begin{itemize}%
        \item Responsibilities:
            \begin{itemize}%
                \item Overall design of the back-end application and REST API
                \item Managed a 2 member team
                \item Front-end internationalization (Japanese / English)
            \end{itemize}
        \item Technologies used: \lang{Go}, \lang{JavaScript}, \techno{WebSocket}, \techno{PostgreSQL}, \techno{L20n}
    \end{itemize}
\ifdefined\detailed%
% WARV-2
\item Worked on the configuration interface for a \techno{WiMAX} gateway system.
    \begin{itemize}%
        \item Responsibilities:
            \begin{itemize}%
                \item Creation of \lang{XSLT} stylesheets to translate \lang{XML} configuration files to system configuration files
                \item Development of a \techno{Net-SNMP} \lang{Perl} module
                \item Development of monitoring daemons in \lang{Go}
            \end{itemize}
        \item Technologies used: \lang{Go}, \lang{Perl}, \lang{XSLT}, \lang{Bash}, \techno{PostgreSQL}
    \end{itemize}
\fi
% E2E
\item Developed a \techno{LTE-A} (4G) network tester, along with several modules for specific network protocols.
    \begin{itemize}%
        \item Responsibilities:
            \begin{itemize}%
                \item Development of network protocol modules used for network testing (\techno{HTTP}, \techno{FTP}, \techno{ICMP}, \techno{SIP}, \techno{RTP})
            \end{itemize}
        \item Technologies used: \lang{C++}, \techno{Boost}
    \end{itemize}
% AMP
\item Created an automated system for attendance management using \techno{FeliCa} tags.
    \begin{itemize}%
        \item Responsibilities:
            \begin{itemize}%
                \item Overall design of the project
                \item Development of an NFC tag scanner to be used with Suica and other FeliCa compatible devices
                \item Development of a web interface allowing timetable viewing and administration
                \item Front-end internationalization (Japanese / English / French)
            \end{itemize}
        \item \lang{Python}, \lang{JavaScript}, \techno{PostgreSQL}, \techno{NFC}
    \end{itemize}
\ifdefined\detailed%
% 10G
\item Developed a traffic analysis tool and configuration interface for 10G networks.
    \begin{itemize}%
        \item Responsibilities:
            \begin{itemize}%
                \item Development of several \lang{Bash} script to manage Linux configuration files
                \item Creation of a communication API using \lang{PHP} and \lang{JavaScript}
                \item Communication with software engineer from partner company
            \end{itemize}
        \item Technologies used: \lang{PHP}, \lang{Bash}, \lang{JavaScript}, \techno{MySQL}, \techno{jQuery}
    \end{itemize}
\fi
\item Technical environment (general):
    \begin{itemize}%
        \item \techno{CentOS}, \techno{Ubuntu}, \techno{Raspbian}, \techno{Git}, \techno{Mercurial}, \techno{Jenkins}, \techno{Redmine}
    \end{itemize}
\end{itemize}
}

\ifdefined\detailed%
\cventry{2010 (April--June)}{Internship}{CEA Valduc}{}{}{Automated the testing process of an internal web application using \techno{Selenium}.\newline{}% % chktex 8
\begin{itemize}%
  \item Technologies: \techno{RHEL}, \lang{PHP}, \lang{JavaScript}, \techno{Selenium}
\end{itemize}}
\fi


\section{Education}
\cventry{2011--2012}{Yoshida Institute of Japanese Language}{Shinjuku}{Tokyo}{}{Japanese language course.}
\cventry{2008--2010}{DUT Informatique}{Burgundy University}{Dijon}{}{2 years technical degree in computer science and software engineering.}
%\cventry{2005--2008}{Baccalauréat STG}{Lycée Le Castel}{Dijon}{\textit{Mention bien}}{}
\cventry{2005--2008}{Baccalauréat STG}{Lycée Le Castel}{Dijon}{}{French high-school diploma.}

\ifdefined\detailed\else
\clearpage
\fi

\section{Languages}
\cvitemwithcomment{French}{Native}{}
\cvitemwithcomment{English}{Business}{}
\cvitemwithcomment{Japanese}{Business}{JLPT N2 obtained in 2012}

% TODO: Différencier entre les niveaux de compétences (ex: expert en gras, proficient en normal)
\section{Computer skills}
\subsection{Programming Languages}
\cvitem{General}{\lang{C}, \lang{C++}, \lang{Java}}
\cvitem{Server Side}{\lang{Erlang}, \lang{Python}, \lang{Ruby}, \lang{Go}, \lang{Node.js}}
\subsection{Technologies}
\cvitem{Operating Systems}{\techno{RHEL}, \techno{CentOS}, \techno{Debian}, \techno{Ubuntu}}
\cvitem{Database}{\techno{Amazon RDS}, \techno{PostgreSQL}, \techno{MySQL}}
% \cvitem{Frameworks}{\techno{RoR}, \techno{Hadoop}}
\cvitem{Mobile}{\techno{Android}, \techno{Kotlin}}
\cvitem{Messaging Protocols}{\techno{AMQP}, \techno{MQTT}, \techno{XMPP}}
% \cvitem{Testing}{\techno{JUnit}, \techno{PHPUnit}, \techno{Rspec}, \techno{Python unittest}, \techno{Selenium}}
\cvitem{Project Management}{\techno{Jira}, \techno{Redmine}}
\cvitem{CI/CD}{\techno{Bitbucket Pipelines}, \techno{AWS CodeDeploy}, \techno{Jenkins}}
\cvitem{Version Control}{\techno{Git}, \techno{Mercurial}, \techno{Subversion}}


\section{Interests}
% TODO: Parler de la trad ?
\cvitem{Computer Science}{Reading about and learning new technologies.}
\cvitem{General}{Watching movies, video games, cycling, etc.}

\section{Miscellaneous}
% \cvlistitem{French and Japanese driving licence}
\cvitem{}{French and Japanese driving licence}

% \section{References}
% \begin{cvcolumns}
%   \cvcolumn{Category 1}{\begin{itemize}\item Person 1\item Person 2\item Person 3\end{itemize}}
%   \cvcolumn{Category 2}{Amongst others:\begin{itemize}\item Person 1, and\item Person 2\end{itemize} (more upon request)}
%   \cvcolumn[0.5]{All the rest \& some more}{\textit{That} person, and \textbf{those} also (all available upon request).}
% \end{cvcolumns}

\end{document}
